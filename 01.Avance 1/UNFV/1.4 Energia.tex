Mencionaremos, por último, que el problema de la catenaria puede abordarse desde un punto de vista completamente distinto, partiendo del hecho de que la catenaria debe ser la curva que une dos puntos con una longitud dad con la mínima energía potencial. La energía potencial de una curva es:\footcite{Car12}
\[E(y)= \int_{0}^{L} gy(s)\rho \cdot ds=\rho g \int_{x_{0}}^{x_{1}} y(x) \sqrt{1+y^{'}(x)^{2}} \cdot dx\]
de donde obtenemos que:
\[y=\frac{h}{2}+\frac{a}{\gamma}(\cosh(\gamma \frac{x}{a} +C_{2})-\cosh(C_{2})\cosh(\gamma) )\]
siendo los puntos extremos de la catenaria $(-a,0)$ y $(a,h)$


