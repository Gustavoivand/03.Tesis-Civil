\section{Introducción}
\begin{frame}{Introducción}

    \justify
    
    El presente trabajo tiene como finalidad resumir los tres primeros capítulos
    del curso de Master ofrecido por ANSYS en Simulación Numérica en Ingeniería.
    En ese sentido es importante agregar que estos tres primers capítulos cubren
    3 aspectos:
    \begin{itemize}
        \item Prefacio.
        \item Diseño y Optimización de Ingeniería.
        \item Métodos de optimización del tipo gradiente.
    \end{itemize}
    %No voy a detenerme en el prefacio, pero sí voy a hacer algunas apreciaciones.
    %el objetivo del curso en general es ofrecer las herramientas necesarias_
    %a los ingenieros para afrontar problemas de optimización.
    
    %Pero ¿Como entendemos un problema de optimización?

    %Desde que se desarrolló la mecánica de materiales valiéndose del cálculo
    %como herramienta de apoyo hasta esta fecha han pasado más de 300 años.

    %Actualmente se han desarrollado modelos matemáticos complejos que
    %describen de manera muy aproximada el comportamiento de los materiales_
    %de construcción que utilizamos, ese no es el alcance de mi investigación
    %pero sí el de mis compañeros en este grupo, por ejemplo.
    %Pero aunado a la complejidad mecánica de un problema dentro de la ingeniería
    %civil, tenemos que analizar muchas veces también factores financieros.
    %Los cuales a veces no tienen una relación obvia con la solución
    %física a la que podamos llegar.

    %En ese sentido la idea detrás del diseño de optimización
    %Es involucrar todos las variables de decisión dentro de nuestra 
    %función de diseño.

    %
\end{frame}

    

