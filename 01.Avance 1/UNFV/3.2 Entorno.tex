\begin{columns}
    \begin{column}{0.47\textwidth}
        \begin{itemize}
            \item Analizamos la función $f(x,y)$ en $\color{blue}P(x_{0},y_{0})=P(0.63;0)$, y obtenemos:
                    \\
                    $\color{blue}f(x_{0},y_{0})=-0.47247$
            \item Analizamos la función $f(x,y)$ en $P_{1,2}(x_{0}-0.1,y_{0} \pm 0.1)=P_{1,2}(0.53; \pm 0.1)$, y obtenemos:
                    \\
                    $f(x_{0}-0.1,y_{0} \pm 0.1)=-0.45$
            \item De lo analizado podemos observar que $P$ es un punto más bajo que $P_{1,2}$, podemos afirmar parcialmente que $P$ es punto mínimo de la función $f$.
        \end{itemize}
    \end{column}
    \begin{column}{0.47\textwidth}
        Analizamos en el plano $x=x_{0}$:
        \begin{tikzpicture}[samples=50]
            \begin{axis}[
                samples=30,axis lines=middle, axis on top, 
                view={60}{45},  
                xmin=-1,
                xmax=1,
                ymin=-1,
                ymax=1,       
                miter limit=1,scale=0.75]
                
                \addplot3[ smooth,     
                    surf,
                    faceted color=gray,
                    line width=0.0001pt, 
                    fill=white, 
                    domain=-1:0.62996,
                    y domain = -1:1,
                    samples = 20,
                    samples y = 20]
                    {H(\x,\y)};
                    
                \addplot3[y domain=-1:1,
                    samples y = 50, samples= 0, red, thick] 
                    ({0.62996},{y},{H(0.62996,y)});
                    
                \addplot3[y domain=-.1:.1,
                    samples y = 10, samples= 0, blue, thick] 
                    ({0.52996},{y},{H(0.52996,y)});
                \addplot3[domain=0.52996:0.62996,
                    samples = 10, samples y = 0, blue, thick] 
                    ({x},{0.1},{H(x,0.1)});  
                    
                \addplot3[domain=0.52996:0.62996,
                    samples = 10, samples y = 0, blue, thick] 
                    ({x},{-0.1},{H(x,-0.1)});  
                
            \end{axis}
            
        \end{tikzpicture} 
    \end{column}
    
\end{columns}

\framebreak

\begin{columns}
    \begin{column}{0.47\textwidth}
        Analizamos en el plano $y=y_{0}$:
        \begin{tikzpicture}[samples=50]
            \begin{axis}[
                samples=30,axis lines=middle, axis on top, 
                view={310}{50},  
                xmin=-0.4,
                xmax=1.6,
                ymin=-1,
                ymax=1,       
                miter limit=1,scale=0.75]
                
                \addplot3[ smooth,     
                    surf,
                    faceted color=gray,
                    line width=0.0001pt, 
                    fill=white, 
                    domain=-0.4:0.62996,
                    y domain = 0:1,
                    samples = 10,
                    samples y = 10]
                    {H(\x,\y)};
                \addplot3[ smooth,     
                    surf,
                    faceted color=gray,
                    line width=0.0001pt, 
                    fill=white, 
                    domain=0.62996:1.5,
                    y domain = -1:1,
                    samples = 10,
                    samples y = 19]
                    {H(\x,\y)};
                    
                \addplot3[domain=-0.4:1.5,
                    samples y = 0, samples= 50, red, thick] 
                    ({x},{0},{H(x,0)});
                    
                \addplot3[y domain=-.1:.1,
                    samples y = 10, samples= 0, blue, thick] 
                    ({0.72996},{y},{H(0.72996,y)});
                \addplot3[y domain=0:.1,
                    samples y = 10, samples= 0, blue, thick] 
                    ({0.52996},{y},{H(0.52996,y)});
                \addplot3[domain=0.52996:0.72996,
                    samples = 10, samples y = 0, blue, thick] 
                    ({x},{0.1},{H(x,0.1)});  
                    
                \addplot3[domain=0.62996:0.72996,
                    samples = 10, samples y = 0, blue, thick] 
                    ({x},{-0.1},{H(x,-0.1)});  
                
            \end{axis}
            
        \end{tikzpicture} 
    \end{column}
    \begin{column}{0.47\textwidth}
        \begin{itemize}
            \item Recordando que el punto $\color{blue}P(x_{0},y_{0})=P(0.63;0)$ en la función $f(x,y)$ toma el valor de:
                    \\
                    $\color{blue}f(x_{0},y_{0})=-0.47247$
            \item Analizamos la función $f(x,y)$ en $P_{3,4}(x_{0}+0.1,y_{0} \pm 0.1)=P_{3,4}(0.73; \pm 0.1)$, y obtenemos:
                    \\
                    $f(x_{0}+0.1,y_{0} \pm 0.1)=-0.446$
            \item De lo analizado podemos observar que $P$ es un punto más bajo que $P_{1,2,3,4}$, podemos afirmar que $P$ es punto mínimo de la función $f$.
        \end{itemize}
    \end{column}
\end{columns}


    

