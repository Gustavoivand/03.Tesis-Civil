\justify
Para aproximarnos a un punto mínimo en la función $f$ a través del método de Newton asumimos un valor donde esperamos que exista un punto mínimo, por ejemplo para nuestro caso usaremos el punto $K(x_{k},y_{k})=(0.5,0.5)$ entonces calculamos un punto más próximo al mínimo esperado al que llamaremos $K+1$, con la siguiente expresión:
\[X_{K+1}=X_{K}-H^{-1}_{f}(X_{K}) * \nabla f(X_{K}) \]
\flushleft
donde:
\begin{itemize}
    \item $X_{K}$: es el punto $K$ expresado vectorialmente, es decir, en nuestro caso $X_{K}=\begin{bmatrix}
        x_{k} \\
        y_{k}
    \end{bmatrix}=
    \begin{bmatrix}
        0.5 \\
        0.5
    \end{bmatrix}$.
    \item $H^{-1}_{f}(X_{K})$: es  la inversa del Hessiano de la función $f$ evaluada en el punto $K$.
    \item $\nabla f(X_{K})$: es la gradiente de la función $f$ evaluada en el punto $K$.
\end{itemize}

\framebreak

cómo vimos anteriormente el hessiano de la funcion $f$, está definido por:
\[H_{f}=\begin{bmatrix}
  f_{xx}(x,y) & f_{xy}(x,y)\\
 f_{xy}(x,y) & f_{yy}(x,y)
\end{bmatrix}=
\begin{bmatrix}
  12x^{2} & 0\\
 0 & 12y^{2}
\end{bmatrix}\]
y la gradiente no es mas qué:
\[\nabla f(X) = \begin{bmatrix}
    f_{x}(x,y)\\
    f_{y}(x,y)
\end{bmatrix}
=\begin{bmatrix}
    4x^{3}-1\\
    4y^{3}
\end{bmatrix}\]
Es decir calcularemos, para nuestro caso particular:
\[X_{K+1}=\begin{bmatrix}
        x_{k} \\
        y_{k}
    \end{bmatrix}-\begin{bmatrix}
  12x^{2} & 0\\
 0 & 12y^{2}
\end{bmatrix}^{-1} * \begin{bmatrix}
    4x^{3}-1\\
    4y^{3}
\end{bmatrix}\]

\framebreak

\begin{itemize}
    \item Realizamos la {\color{blue}primeraa iteración} haciendo $
    \begin{bmatrix}
        x_{k} \\
        y_{k}
    \end{bmatrix}=
    \begin{bmatrix}
        0.5 \\
        0.5
    \end{bmatrix}$, de donde obtenemos:\\
    \scalebox{0.75}{$X_{K+1}=\begin{bmatrix}
        0.5 \\
        0.5
    \end{bmatrix}-\begin{bmatrix}
  12(0.5)^{2} & 0\\
 0 & 12(0.5)^{2}
\end{bmatrix}^{-1} * \begin{bmatrix}
    4(0.5)^{3}-1\\
    4(0.5)^{3}
\end{bmatrix}=\begin{bmatrix}
        0.6666\\
        0.3333
\end{bmatrix}
$}
\item Realizamos la {\color{blue}segunda iteración} haciendo $
    \begin{bmatrix}
        x_{k} \\
        y_{k}
    \end{bmatrix}=
    \begin{bmatrix}
        0.666 \\
        0.333
    \end{bmatrix}$, de donde obtenemos:\\
    \scalebox{0.75}{$X_{K+1}=\begin{bmatrix}
        0.666 \\
        0.3333
    \end{bmatrix}-\begin{bmatrix}
  12(0.666)^{2} & 0\\
 0 & 12(0.333)^{2}
\end{bmatrix}^{-1} * \begin{bmatrix}
    4(0.666)^{3}-1\\
    4(0.333)^{3}
\end{bmatrix}=\begin{bmatrix}
        0.6319\\
        0.2222
\end{bmatrix}
$}
\item Realizamos la {\color{blue}tercera iteración} haciendo $
    \begin{bmatrix}
        x_{k} \\
        y_{k}
    \end{bmatrix}=
    \begin{bmatrix}
        0.6319 \\
        0.2222
    \end{bmatrix}$, de donde obtenemos:\\
    \scalebox{0.75}{$X_{K+1}=\begin{bmatrix}
        0.6319 \\
        0.2222
    \end{bmatrix}-\begin{bmatrix}
  12(0.6319)^{2} & 0\\
 0 & 12(0.2222)^{2}
\end{bmatrix}^{-1} * \begin{bmatrix}
    4(0.6319)^{3}-1\\
    4(0.2222)^{3}
\end{bmatrix}=\begin{bmatrix}
        0.63\\
        0.1481
\end{bmatrix}
$}
\item Realizamos la {\color{blue}cuarta iteración} haciendo $
    \begin{bmatrix}
        x_{k} \\
        y_{k}
    \end{bmatrix}=
    \begin{bmatrix}
        0.63 \\
        0.1481
    \end{bmatrix}$, de donde obtenemos:\\
    \scalebox{0.75}{$X_{K+1}=\begin{bmatrix}
        0.63 \\
        0.1481
    \end{bmatrix}-\begin{bmatrix}
  12(0.63)^{2} & 0\\
 0 & 12(0.1481)^{2}
\end{bmatrix}^{-1} * \begin{bmatrix}
    4(0.63)^{3}-1\\
    4(0.1481)^{3}
\end{bmatrix}=\begin{bmatrix}
        0.63\\
        0.0987
\end{bmatrix}
$}
\item Realizamos la {\color{blue}quinta iteración} haciendo $
    \begin{bmatrix}
        x_{k} \\
        y_{k}
    \end{bmatrix}=
    \begin{bmatrix}
        0.63 \\
        0.0987
    \end{bmatrix}$, de donde obtenemos:\\
    \scalebox{0.75}{$X_{K+1}=\begin{bmatrix}
        0.63 \\
        0.0987
    \end{bmatrix}-\begin{bmatrix}
  12(0.63)^{2} & 0\\
 0 & 12(0.0987)^{2}
\end{bmatrix}^{-1} * \begin{bmatrix}
    4(0.63)^{3}-1\\
    4(0.0987)^{3}
\end{bmatrix}={\color{red}\begin{bmatrix}
        0.63\\
        0.0658
\end{bmatrix}
}$}
\end{itemize}
Como hemos visto, el método lleva a la convergencia del punto inicial elegido hacia el punto crítico que calculamos analíticamente.

\framebreak
Finalmente, para saber si el punto encontrado es un máximo o un mínimo local, tenemos que evaluar $f_{xx}$, al igual que en el método analítico, si $f_{xx}<0$, entonces estamos hablando de un máximo local, si, por el contrario $f_{xx}>0$, entonces el punto analizado es un mínimo local.